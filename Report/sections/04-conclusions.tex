% 04-conclusions.tex

% Section Title
\section{CONCLUSIONS} \label{sec:conclusions}

    This research compared GNSS performance on a commercial smart-phone under static, dynamic, and spoofed scenarios based on raw measurements and post-processing in MATLAB. Under the static scenario, the device had consistent position estimates, low clock drift, and negligible variation in pseudorange and speed. The dynamic scenario, on the other hand, demonstrated higher noise in position, pseudor-range rates, and clock bias—predictable effects of motion and satellite geometry variations.
    To further assess receiver robustness, we introduced an interference scenario by shielding the device with aluminum foil and placing it near two actively transmitting smartphones. This setup caused a noticeable drop in signal quality, a reduction in the number of tracked satellites, and significant degradation in position accuracy. Despite the device remaining stationary, the solution drifted tens of meters, revealing the sensitivity of smartphone GNSS to even passive signal obstruction and nearby RF noise.
    The spoofing tests indicated that even basic, software-level injec-arise in the form of taking on false coordinates by the GNSS solver, as it settles on a consistent but incorrect position with little or no impact on first-order signal measurements like carrier-to-noise ratio or HDOP. This highlights a significant vulnerability: smartphone GNSS receivers can be fooled without revealing obvious degradations in signal quality.