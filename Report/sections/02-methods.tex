% 02-methods.tex

% Section Title
\section{METHODS} \label{sec:methods}

    % In prose, describe the end-to-end procedure
    % - Devices and software: handset model, app, toolbox
    % - Data collection: where, how long, static vs. moving
    % - Spoofed-inputs: false position and delay parameters
    % - Optional interference scenario
    % - Processing: filtering criteria and solver workflow

    \subsection{Devices and Software}
    
      We used a Samsung Galaxy A51 smartphone running Android 11 with GNSS Logger v3.1.0.4 to capture raw measurements. 
      Data processing and analysis were performed in MATLAB R2024b using Google's open-source \texttt{gps-measurement-tools} library.
    
      \vspace{-0.2cm}

    \subsection{Data Collection}
    
      Two datasets were acquired on 3 May 2025 under cloudy conditions:

      \begin{itemize}
        \item \textbf{Static (Monte dei Cappuccini):} Device fixed by hand on a rooftop at latitude 45.059888°, longitude 7.697348°; logging ran from 10:35:20 for 5 minutes.
        \item \textbf{Dynamic (Tram Route):} Handheld during a tram ride between start point 45.0707622°, 7.6850644° and end point 45.063912°, 7.696680°; logging ran from 10:00:21 for 5 minutes.
      \end{itemize}

      \vspace{-0.2cm}

    \subsection{Processing Pipeline}
    
      Raw GNSS logs were first filtered to exclude observations below a carrier-to-noise ratio of 25 dB-Hz and satellite elevation below 15°. 
      We extracted pseudorange and Doppler measurements per epoch, then computed a weighted least-squares (WLS) position solution. 
      For spoof-input evaluation, the same static dataset was reprocessed by specifying a false reference location (\texttt{spoof.position}) and an artificial timing delay (\texttt{spoof.delay}), allowing comparison of the shifted navigation output against the baseline.
